%PREAMBULO MEMORIA
\documentclass[a4paper, 12pt]{book}
\usepackage[a4paper, left=2.5cm, right=2.5cm, top=2cm, bottom=3cm]{geometry}
\usepackage[utf8]{inputenc} %Para introducir tildes
\usepackage{url} %Para introducir urls
\usepackage{tocbibind} %Bibliografía en el indice
\usepackage[spanish]{babel}
\usepackage[toc]{blindtext}
\usepackage{tocbibind}
\usepackage{float} %Para posicionar figuras
\usepackage{graphicx}
\usepackage{fancyhdr}
\usepackage{hyperref}
\usepackage{titlesec, blindtext, color}
%%Estilo de fuente
\usepackage{lmodern}
\usepackage[OT1]{fontenc}


% Personalización de Formato de Capítulo
\usepackage[Lenny]{fncychap}

% Personalización del tipo de fuente



\title{Memoria del Proyecto}
\author{Javier Fernández Morata}

\renewcommand{\baselinestretch}{1.5}

\begin{document}

\renewcommand{\refname}{Bibliografía}
%%%%%%%%%%%%%%%%%%%%%%%%%%%%%%%%%%%%%%%%%%%%%%%%%%%%%%%%%%%%%%%%%%%%%%%%%%%%%%%
% PORTADA
%%%%%%%%%%%%%%%%%%%%%%%%%%%%%%%%%%%%%%%%%%%%%%%%%%%%%%%%%%%%%%%%%%%%%%%%%%%%%%%
\begin{titlepage}
\begin{center}
\begin{tabular}[c]{c c}
\includegraphics[scale=0.25]{img/logo_vect.png} &
\begin{tabular}[b]{l}
\Huge
\textsf{UNIVERSIDAD} \\
\Huge
\textsf{REY JUAN CARLOS} \\
\end{tabular}
\\
\end{tabular}

\vspace{3cm}
\Large
INGENIERÍA EN TECNOLOGÍAS DE LA TELECOMUNICACIÓN

\vspace{0.4cm}
\Large
Curso Acádemico 2018/2019

\vspace{0.8cm}
Trabajo Fin de Grado

\vspace{2.5cm}
\Large
{{ TÍTULO DE TRABAJO }}

\vspace{4cm}
\large
Autor : Javier Fern\'andez Morata \\
Tutor : Dr. Felipe Ortega
\end{center}
\end{titlepage}

\newpage
\thispagestyle{empty}
%%%%%%%%%%%%%%%%%%%%%%%%%%%%%%%%%%%%%%%%%%%%%%%%%%%%%%%%%%%%%%%%%%%%%%%%%%%%%%%
% PARA FIRMAR
%%%%%%%%%%%%%%%%%%%%%%%%%%%%%%%%%%%%%%%%%%%%%%%%%%%%%%%%%%%%%%%%%%%%%%%%%%%%%%%
\clearpage
\pagenumbering{gobble}
\chapter*{}

\vspace{-4cm}
\begin{center}
  \large
  \textbf{Trabajo Fin de Grado}

  \vspace{1cm}
  \large
  \{\{Título del Trabjo con Letras Capitales para Sustantivos y Adjetivos\}\}

  \vspace{1cm}
  \large
  \textbf{Autor :} Javier Fernández Morata \\
  \textbf{Tutor :} Dr. Felipe Ortega
\end{center}

\vspace{1cm}
La defensa del presente Proyecto Fin de Carrera se realiza el día \qquad$\;\,$ de \qquad\qquad\qquad\qquad \newline de 20XX, siendo calificada por el siguiente tribunal:

\vspace{0.5cm}
\textbf{Presidente:}

\vspace{1.2cm}
\textbf{Secretario:}

\vspace{1.2cm}
\textbf{Vocal:}

\vspace{1.2cm}
y habiendo obtenido la siguiente calificación:

\vspace{1cm}
\textbf{Calificación:}

\vspace{1cm}
\begin{flushright} Fuenlabrada, a \qquad$\;\,$ de
  \qquad\qquad\qquad\qquad de 20XX
\end{flushright}


%%%%%%%%%%%%%%%%%%%%%%%%%%%%%%%%%%%%%%%%%%%%%%%%%%%%%%%%%%%%%%%%%%%%%%%%%%%%%%%
% DEDICATORIA
%%%%%%%%%%%%%%%%%%%%%%%%%%%%%%%%%%%%%%%%%%%%%%%%%%%%%%%%%%%%%%%%%%%%%%%%%%%%%%%
\chapter*{}
\begin{flushright}
  \textit{\_Escribimos Dedicatoria\_}
\end{flushright}


%%%%%%%%%%%%%%%%%%%%%%%%%%%%%%%%%%%%%%%%%%%%%%%%%%%%%%%%%%%%%%%%%%%%%%%%%%%%%%%
% AGRADECIMIENTOS
%%%%%%%%%%%%%%%%%%%%%%%%%%%%%%%%%%%%%%%%%%%%%%%%%%%%%%%%%%%%%%%%%%%%%%%%%%%%%%%
\chapter*{Agradecimientos}

[ AQUÍ VAN LOS AGRADECIMIENTOS ]

%%%%%%%%%%%%%%%%%%%%%%%%%%%%%%%%%%%%%%%%%%%%%%%%%%%%%%%%%%%%%%%%%%%%%%%%%%%%%%%
% RESUMEN
%%%%%%%%%%%%%%%%%%%%%%%%%%%%%%%%%%%%%%%%%%%%%%%%%%%%%%%%%%%%%%%%%%%%%%%%%%%%%%%
\chapter*{Resumen}

Aquí viene un resumen del proyecto.Ha de constar de tres o cuatro párrafors, donde se presente de manera clara y concisa de qué va el proyecto.
Han de quedar respondidas las siguientes preguntas:
\begin{itemize}
  \item ¿De qué va este proyecto? ¿Cuál es el objetivo principal?
  \item ¿Cómo se ha realizado? ¿Qué tecnologías están involucradas?
  \item ¿En qué contexto se ha realizado el Proyecto? Es un proyecto dentro de un marco general?
\end{itemize}

 %%%%%%%%%%%%%%%%%%%%%%%%%%%%%%%%%%%%%%%%%%%%%%%%%%%%%%%%%%%%%%%%%%%%%%%%%%%%%%%
 % ÍNDICE
 %%%%%%%%%%%%%%%%%%%%%%%%%%%%%%%%%%%%%%%%%%%%%%%%%%%%%%%%%%%%%%%%%%%%%%%%%%%%%%%
%% contenidos
\tableofcontents

%%%%%%%%%%%%%%%%%%%%%%%%%%%%%%%%%%%%%%%%%%%%%%%%%%%%%%%%%%%%%%%%%%%%%%%%%%%%%%%
% CAPÍTULO 1 - INTRODUCCIÓN Y OBJETIVOS
%%%%%%%%%%%%%%%%%%%%%%%%%%%%%%%%%%%%%%%%%%%%%%%%%%%%%%%%%%%%%%%%%%%%%%%%%%%%%%%

\chapter{Introducción y objetivos}
\label{chap:intro}
\pagenumbering{arabic}
En este proyecto tratamos la importancia de seguir unas buenas pautas para la gestión y distribución de código. Donde de todas ellas ahondaremos en la creación de documentación web, donde nos ayudaremos de varias tecnologías. \\
El objetivo por tanto de este proyecto, es presentar las pautas a seguir para facilitar las contribuciones futuras en proyectos de software libre y mostrar una de ellas como, la documentación. \\
Por tanto, la intención de este capítulo es mostrar el contexto, la motivación que  me ha llevado para realizar dicho proyecto, objetivos de este, y la estructura que vamos a seguir para mostrar lo realizado.

\section{Contexto}
\label{sec:contex}
A los largos de los años, el software ha pasado por varias etapas en cuanto a su privatización. Antes del \emph{boom} de la informática, los que hacían uso de ella, compartían \emph{software} sin ninguna restricción, pero cuando llegaron los años 80 las compañías que vendían las computadoras comercializaban estas usando sistemas operativos privados, forzando al usuario a aceptar restricciones legales de modificar dicho \emph{software}. \\
Por estos años y debido a un error con un dispositivo, Richard Stallman fundó el proyecto GNU e introdujo la definición de \emph{software libre}. ¿Qué es software libre? Software libre es la cuestión de libertad de ejecutar, copiar, distribuir, estudiar, cambiar y mejorar el software. \\ \\ \\ 
Es decir, significa que los usuarios de dicho software disponemos de las cuatro libertades esenciales: 
\begin{itemize}
    \item Libertad 0, libertad para ejecutar el \emph{software} con cualquier propósito
    \item Libertad 1, acceso al código fuente; por lo que podemos modificar dicho \emph{software} para hacer lo que el usuario quiera
    \item Libertad 2, redistribuir copias 
    \item Libertad 3, redistribución de versiones modificadas.
\end{itemize}

Y por tanto, uno de los objetivos de este proyecto es facilitar las contribuciones futuras para unos módulos sobre ingeniería biomédica desarrollados en la Universidad Rey Juan Carlos I, donde para ello, se creara una documentación web, usando tecnologías como Jekyll, que es un generador de contenido estático, o herramientas como ReadTheDocs o Sphinx, que facilitan la creación de documentación de \emph{software} 


\section{Motivación Personal}
\label{sec:mot}
Desde que empecé el instituto siempre me he decantado por una manera de trabajar más práctica que teórica. Si algo tengo que destacar de la carrera es el descubrir la programación, algo que sabía que era, pero que nunca había practicado, es decir, no había programado nunca un "Hola Mundo". \\
Una de las asignaturas que más me entusiasmo cursar, fue la de \emph{"Servicios y Aplicaciones Telemáticas"}, en la que desarrollamos aplicaciones web a través del \emph{framework} Django. En este proyecto además de un desarrollo web, con distintas tecnologías que utilice en dicha asignatura, hay otro objetivo, que es el más interesante para mi, que es seguir unas pautas de escribir código para proyectos de \emph{Software libre}.
\newpage

\section{Objetivos}
\label{sec:objetivos}
El objetivo principal, por tanto, es la creación de un sitio web, para dar a conocer la librería python \emph{pyCardio}. Dicha librería ha sido desarrollada por varios alumnos y profesores de esta Universidad. Donde con esta web lo que se propone es dar a conocer los módulos de la librería y mostrar su documentación. \\
Otro propósito de dicha web, al ser un proyecto de \emph{Software libre}, es facilitar la contribución al código de este paquete. Para alcanzar dichos fines, han sido abordados las siguientes metas secundarias :
\begin{itemize}
    \item Aprendizaje de nuevas tecnologías usadas en el proyecto.
    \item Publicar contenido mediante Jekyll 
    \item Crear la docWeb en GitHub Pages
    \item Aprender a usar Liquid, para optimizar el desarrollo del contenido
\end{itemize}

\section{Estructura de la memoria}
\label{sec:estruc}
Para finalizar la introducción, se explica como va a ir estructurada la memoria seguida de un breve resumen del contenido que trata cada sección: 
\begin{enumerate}
    \item En el capítulo uno, como ya hemos visto, se introduce el por qué de este proyecto así como los objetivos que se tratan en él.
    \item En el segundo, \emph{Soluciones tecnológicas}, se explican las tecnologías usadas en el proyecto tanto en la generación del sitio web, como de gestión y documentación de proyectos Python.
    \item En el tercer capítulo, \emph{Gestión de código}, trataremos sobre la gestión de código en python y sobre el paquete que centramos el objetivo de este proyecto \emph{pyCardio}.
    \item En este cuarto capítulo, \emph{Propuesta del diseño web del proyecto}, presentamos la web del proyecto donde se incluirá la documentación del módulo, guía de instalación, así como una guía para futuras contribuciones.
    \item En el quinto y último capítulo, \emph{Conclusiones y trabajo futuro}, exponemos las conclusiones sacadas así como el trabajo que se debería realizar tras la finalización de este mismo.
\end{enumerate}

%%%%%%%%%%%%%%%%%%%%%%%%%%%%%%%%%%%%%%%%%%%%%%%%%%%%%%%%%%%%%%%%%%%%%%%%%%%%%%%
% CAPÍTULO 2 - ARQUITECTURA GENERAL
%%%%%%%%%%%%%%%%%%%%%%%%%%%%%%%%%%%%%%%%%%%%%%%%%%%%%%%%%%%%%%%%%%%%%%%%%%%%%%%
\chapter{Soluciones tecnológicas}
\label{chap:Arqui}
Este capítulo constará de dos partes. Una primera donde se expone todas las tecnologías utilizadas en la creación del sitio web del proyecto, y una segunda donde hablamos sobre las tecnologías de gestión de proyecto y documentación en Python.


\section{Arquitectura general}a
\label{sec:arqui}
[[  PREGUNTAR TUTOR, COMO QUEREMOS REDACTAR ESTA PARTE]

\section{Control de versiones con Git. GitHub}
\label{sec:git}
Este proyecto al ser un trabajo de \emph{software libre}, significa que el código sufrirá cambios de distintos autores, por lo que es necesario un sistema de control de versiones. Para que un sistema sea de control de versiones debe proporcionar un mecanismo de almacenamiento de los elementos que se desea gestionar, posibilidad de realizar cambios sobre los elementos almacenados y un registro histórico de las acciones realizadas con cada elemento o conjunto de elementos. \\
El sistema que se ha escogido debido a nuestra arquitectura de almacenamiento de código y por su seguridad, comodidad y velocidad, es \emph{Git}. Una gran diferencia de \emph{Git} respecto a los demás sistemas de control de versiones es la manera de llevar el registro de cambios sobre sus datos, mientras que la mayoría de sistemas almacenan la información como una lista de cambios, en \emph{Git}, cada vez que se realiza un cambio sobre un archivo \emph{Git} hace una instantánea, y guarda una referencia sobre el archivo sin estos cambios. Para ser mas eficiente, si los archivos no han sido modificados, \emph{Git} no almacena el archivo de nuevo. % esta distinción influye en uno de los mayores beneficios de \emph{Git}, las ramificaciones, tema que trataremos a continuación, viendo como funciona este sistema de control de versiones. \\
\textbf{¿Cómo funciona \emph{Git}?}\\
Todo el funcionamiento de este sistema de control de versiones es mayoritariamente local, es decir, no se necesita información de ningún servidor haciendo así que este sistema sea muy rápido respecto a otros que necesitan información remota para sus operaciones. \\
\emph{Git} tiene tres estados para los archivos que trabajamos:
\begin{itemize}
    \item \textbf{Committed}: Este estado indica que los archivos que han sido modificados, se han almacenado de manera segura en la base de datos local.
    \item \textbf{Modified}: Indica que los archivos que han sido modificados, no han sido confirmados, es decir, los archivos no se han almacenado de forma segura en la base de datos local.
    \item \textbf{Staged}: Estado que marca archivos modificados para la siguiente confirmación. 
\end{itemize}
Estos tres estados nos lleva a diferenciar las siguientes áreas de trabajo: 
\begin{itemize}
    \item \textbf{Working Directory}: Este área es donde obtenemos la copia del proyecto, lista para usarse o modificarse, sin influir en la copia original.
    \item \textbf{Staged Area}: Área en la que se recoge los cambios que van a realizarse sobre los archivos antes de confirmarse.
    \item \textbf{Git Directory}: Estado de los archivos originales del proyecto, incluidos metadatos. Cuando se confirma un archivo, \emph{Git} toma los cambios del área de preparación (Staged Área) y almacena los cambios nuevos en el directorio \emph{Git}.
\end{itemize}
Por tanto, el flujo que hay que seguir para trabajar con este sistema de control de versiones es el representado en la figura ((INTRODUCIMOS IMAGEN-> AQUI REFERENCIA)), donde como vemos tras modificar una serie de archivos en el directorio de trabajo preparamos los archivos añadiéndolos al staged area, se confirman los cambios realizados sobre estos, añadiendo así las instantáneas (copias de los archivos con los cambios confirmados) en el directorio \emph{Git}. \\
{{EMPEZAMOS CON GITHUB}}

%%%%%%%%%%%%%%%%%%%%%%%%%%%%%%%%%%%%%%%%%%%%%%%%%%%%%%%%%%%%%%%%%%%%%%%%%%%%%%%
% DUDAS
%%%%%%%%%%%%%%%%%%%%%%%%%%%%%%%%%%%%%%%%%%%%%%%%%%%%%%%%%%%%%%%%%%%%%%%%%%%%%%%
\chapter*{Dudas}
\begin{enumerate}
    \item ¿Cuantos objetivos debo poner?¿Los puestos son suficientes?
    \item ¿La figura para la Arquitectura General, es solo sobre el procedimiento para crear la web, o solo para la DOC?
    \item ¿En \ref{sec:git} hablamos de las ramificaciones?. Si es así quitar comentario.
\end{enumerate}

%%%%%%%%%%%%%%%%%%%%%%%%%%%%%%%%%%%%%%%%%%%%%%%%%%%%%%%%%%%%%%%%%%%%%%%%%%%%%%%
% NOTAS
%%%%%%%%%%%%%%%%%%%%%%%%%%%%%%%%%%%%%%%%%%%%%%%%%%%%%%%%%%%%%%%%%%%%%%%%%%%%%%%
\chapter*{Notas}
\begin{enumerate}
    \item Empezar introducción con una frase del que creo el software Libre
\end{enumerate}


\end{document}
