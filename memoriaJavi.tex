%PREAMBULO MEMORIA
\documentclass[a4paper, 12pt]{book}
\usepackage[a4paper, left=2.5cm, right=2.5cm, top=2cm, bottom=3cm]{geometry}
\usepackage[utf8]{inputenc} %Para introducir tildes
\usepackage{url} %Para introducir urls
\usepackage{tocbibind} %Bibliografía en el indice
\usepackage[spanish]{babel}
\usepackage[toc]{blindtext}
\usepackage{tocbibind}
\usepackage{float} %Para posicionar figuras
\usepackage{graphicx}
\usepackage{fancyhdr}
\usepackage{titlesec, blindtext, color}
%%Estilo de fuente
\usepackage{lmodern}
\usepackage[OT1]{fontenc}


% Personalización de Formato de Capítulo
\usepackage[Lenny]{fncychap}

% Personalización del tipo de fuente



\title{Memoria del Proyecto}
\author{Javier Fernández Morata}

\renewcommand{\baselinestretch}{1.5}

\begin{document}

\renewcommand{\refname}{Bibliografía}
%%%%%%%%%%%%%%%%%%%%%%%%%%%%%%%%%%%%%%%%%%%%%%%%%%%%%%%%%%%%%%%%%%%%%%%%%%%%%%%
% PORTADA
%%%%%%%%%%%%%%%%%%%%%%%%%%%%%%%%%%%%%%%%%%%%%%%%%%%%%%%%%%%%%%%%%%%%%%%%%%%%%%%
\begin{titlepage}
\begin{center}
\begin{tabular}[c]{c c}
\includegraphics[scale=0.25]{img/logo_vect.png} &
\begin{tabular}[b]{l}
\Huge
\textsf{UNIVERSIDAD} \\
\Huge
\textsf{REY JUAN CARLOS} \\
\end{tabular}
\\
\end{tabular}

\vspace{3cm}
\Large
INGENIERÍA EN TECNOLOGÍAS DE LA TELECOMUNICACIÓN

\vspace{0.4cm}
\Large
Curso Acádemico 2018/2019

\vspace{0.8cm}
Trabajo Fin de Grado

\vspace{2.5cm}
\Large
{{ TÍTULO DE TRABAJO }}

\vspace{4cm}
\large
Autor : Javier Fern\'andez Morata \\
Tutor : Dr. Felipe Ortega
\end{center}
\end{titlepage}

\newpage
\thispagestyle{empty}
%%%%%%%%%%%%%%%%%%%%%%%%%%%%%%%%%%%%%%%%%%%%%%%%%%%%%%%%%%%%%%%%%%%%%%%%%%%%%%%
% PARA FIRMAR
%%%%%%%%%%%%%%%%%%%%%%%%%%%%%%%%%%%%%%%%%%%%%%%%%%%%%%%%%%%%%%%%%%%%%%%%%%%%%%%
\clearpage
\pagenumbering{gobble}
\chapter*{}

\vspace{-4cm}
\begin{center}
  \large
  \textbf{Trabajo Fin de Grado}

  \vspace{1cm}
  \large
  \{\{Título del Trabjo con Letras Capitales para Sustantivos y Adjetivos\}\}

  \vspace{1cm}
  \large
  \textbf{Autor :} Javier Fernández Morata \\
  \textbf{Tutor :} Dr. Felipe Ortega
\end{center}

\vspace{1cm}
La defensa del presente Proyecto Fin de Carrera se realiza el día \qquad$\;\,$ de \qquad\qquad\qquad\qquad \newline de 20XX, siendo calificada por el siguiente tribunal:

\vspace{0.5cm}
\textbf{Presidente:}

\vspace{1.2cm}
\textbf{Secretario:}

\vspace{1.2cm}
\textbf{Vocal:}

\vspace{1.2cm}
y habiendo obtenido la siguiente calificación:

\vspace{1cm}
\textbf{Calificación:}

\vspace{1cm}
\begin{flushright} Fuenlabrada, a \qquad$\;\,$ de
  \qquad\qquad\qquad\qquad de 20XX
\end{flushright}


%%%%%%%%%%%%%%%%%%%%%%%%%%%%%%%%%%%%%%%%%%%%%%%%%%%%%%%%%%%%%%%%%%%%%%%%%%%%%%%
% DEDICATORIA
%%%%%%%%%%%%%%%%%%%%%%%%%%%%%%%%%%%%%%%%%%%%%%%%%%%%%%%%%%%%%%%%%%%%%%%%%%%%%%%
\chapter*{}
\begin{flushright}
  \textit{\_Escribimos Dedicatoria\_}
\end{flushright}


%%%%%%%%%%%%%%%%%%%%%%%%%%%%%%%%%%%%%%%%%%%%%%%%%%%%%%%%%%%%%%%%%%%%%%%%%%%%%%%
% AGRADECIMIENTOS
%%%%%%%%%%%%%%%%%%%%%%%%%%%%%%%%%%%%%%%%%%%%%%%%%%%%%%%%%%%%%%%%%%%%%%%%%%%%%%%
\chapter*{Agradecimientos}

[ AQUÍ VAN LOS AGRADECIMIENTOS ]

 %%%%%%%%%%%%%%%%%%%%%%%%%%%%%%%%%%%%%%%%%%%%%%%%%%%%%%%%%%%%%%%%%%%%%%%%%%%%%%%
 % ÍNDICE
 %%%%%%%%%%%%%%%%%%%%%%%%%%%%%%%%%%%%%%%%%%%%%%%%%%%%%%%%%%%%%%%%%%%%%%%%%%%%%%%
%% contenidos
\tableofcontents

%%%%%%%%%%%%%%%%%%%%%%%%%%%%%%%%%%%%%%%%%%%%%%%%%%%%%%%%%%%%%%%%%%%%%%%%%%%%%%%
% CAPÍTULO 1 - INTRODUCCIÓN Y OBJETIVOS
%%%%%%%%%%%%%%%%%%%%%%%%%%%%%%%%%%%%%%%%%%%%%%%%%%%%%%%%%%%%%%%%%%%%%%%%%%%%%%%

\chapter{Introducción y objetivos}
\label{chap:intro}
\pagenumbering{arabic}
En este proyecto tratamos la importancia de seguir unas buenas pautas para paquetes y distribución de código. Donde de todas ellas ahondaremos en la creación de documentación web, donde nos ayudaremos de varias tecnologías. \\
El objetivo por tanto de este proyecto, es presentar las pautas a seguir para facilitar las contribuciones futuras en proyectos de software libre y mostrar una de ellas como, la documentación. \\
Por tanto, la intención de este capítulo es mostrar el contexto, la motivación que  me ha llevado para realizar dicho proyecto, objetivos de este, y la estructura que vamos a seguir para mostrar lo realizado.

\section{Contexto}
\label{sec:contex}
[[ DONDE HABLO SOBRE EL POR QUÉ DEL PROYECTO Y DEBIDO A QUÉ ]]
\section{Motivación Personal}
\label{sec:mot}
[[ COMENTO SOBRE POR QUÉ ME HA LLAMADO LA ATENCIÓN ESTE PROYECTO Y QUE HA SOLAPADO CON ELLO]]
\section{Objetivos}
\label{sec:objetivos}
[[ ENUMERO LOS OBJETIVOS POR UNA LISTA DESORDENADA ]]
\section{Estructura de la memoria}
\label{sec:estruc}
[[ ESTRUCTURA DE LA MEMORIA ]]

[[DEBE OCUPAR 2 PG]]

%%%%%%%%%%%%%%%%%%%%%%%%%%%%%%%%%%%%%%%%%%%%%%%%%%%%%%%%%%%%%%%%%%%%%%%%%%%%%%%
% RESUMEN
%%%%%%%%%%%%%%%%%%%%%%%%%%%%%%%%%%%%%%%%%%%%%%%%%%%%%%%%%%%%%%%%%%%%%%%%%%%%%%%
\chapter*{Resumen}

Aquí viene un resumen del proyecto.Ha de constar de tres o cuatro párrafors, donde se presente de manera clara y concisa de qué va el proyecto.
Han de quedar respondidas las siguientes preguntas:
\begin{itemize}
  \item ¿De qué va este proyecto? ¿Cuál es el objetivo principal?
  \item ¿Cómo se ha realizado? ¿Qué tecnologías están involucradas?
  \item ¿En qué contexto se ha realizado el Proyecto? Es un proyecto dentro de un marco general?
\end{itemize}

\end{document}
