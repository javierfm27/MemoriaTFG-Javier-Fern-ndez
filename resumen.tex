%%%%%%%%%%%%%%%%%%%%%%%%%%%%%%%%%%%%%%%%%%%%%%%%%%%%%%%%%%%%%%%%%%%%%%%%%%%%%%%
% RESUMEN
%%%%%%%%%%%%%%%%%%%%%%%%%%%%%%%%%%%%%%%%%%%%%%%%%%%%%%%%%%%%%%%%%%%%%%%%%%%%%%%
\chapter*{Resumen}

Este proyecto nace a partir del desarrollo de una librería desarrollada por un equipo dentro de la Universidad Rey Juan Carlos. Esta librería es una biblioteca de funciones y clases desarrollada en Python centrada en el ámbito científico de análisis de anomalías del corazón. Al tener esta librería entre manos y tener como meta que sea usada en el desarrollo de aplicaciones científicas es necesaria su distribución, una buena gestión de código y una página web en que facilite el alcance de este paquete. \\

Los objetivos que nos marcamos son llevados a cabo mediante tecnologías como Jekyll, generador de servidores web estáticos, servicios de gestión de código como Travis y librerías de python para realizar su empaquetado y facilitar así la distribución.

La web conseguida se puede encontrar en el siguiente enlace \url{https://javierfm27.github.io/} donde dentro de ella se puede conseguir acceder a todo lo demás, como el código en PyPi, la documentación de las funciones.
