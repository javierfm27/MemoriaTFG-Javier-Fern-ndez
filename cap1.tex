%%%%%%%%%%%%%%%%%%%%%%%%%%%%%%%%%%%%%%%%%%%%%%%%%%%%%%%%%%%%%%%%%%%%%%%%%%%%%%%
% CAPÍTULO 1 - INTRODUCCIÓN Y OBJETIVOS
%%%%%%%%%%%%%%%%%%%%%%%%%%%%%%%%%%%%%%%%%%%%%%%%%%%%%%%%%%%%%%%%%%%%%%%%%%%%%%%
\chapter{Introducción y objetivos}
\label{chap:intro}
\pagenumbering{arabic}
En este proyecto tratamos la importancia de seguir unas buenas pautas para la gestión y distribución de código. Ahondaremos en la creación de documentación web y empaquetado de código para ello nos ayudaremos de varias tecnologías. \\
El objetivo por tanto de este proyecto, es presentar las pautas a seguir para facilitar las contribuciones futuras en proyectos de software libre y mostrarlas. \\
La intención de este capítulo es mostrar el contexto en el que se va a desarrollar el proyecto, la motivación que  me ha llevado para realizarlo, los objetivos de este, y la estructura que vamos a seguir para mostrar lo realizado.

\section{Contexto}
\label{sec:contex}
Actualmente, nos encontramos en la cuarta era del \emph{software}, era en la que el \emph{software} goza de una grata importancia debido al continuo avance tecnológico de cualquier área de negocio. Partiendo de esta situación, podemos distinguir dos corrientes en el desarrollo del \emph{software}: \emph{software} libre y \emph{software} de propietario. La principal diferencia entre estas corrientes es poner a la disposición del usuario el código fuente del \emph{software} que se esta manejando, permitiendo así que la colaboración entre programadores permita mejorar e innovar el \emph{software}, haciendo así una manera más precisa y rápida de llegar a los objetivos de desarrollo. Apoyando por tanto la corriente de desarrollo de \emph{software} libre el seguir una buena estructura de proyecto, documentación, gestión y correcta distribución del código, es de vital importancia para que las contribuciones futuras en un proyecto de desarrollo \emph{software} sean fáciles de gestionar y de incluir en el proyecto original, o directamente fáciles de uso para el programador que maneja el código. \\ 

Por tanto, el principal objetivo es facilitar las contribuciones futuras para una librería sobre ingeniería biomédica desarrollados en la Universidad Rey Juan Carlos, para lo cual, se creara una documentación web, una web asociada a dicho módulo explicando sus funcionalidades y cómo usarlo, empaquetado para su distribución e instalación; usando tecnologías como \emph{Jekyll}, que es un generador de contenido estático, o herramientas como \emph{ReadTheDocs} o \emph{Sphinx}, que facilitan la creación de documentación de \emph{software}. También apoyaremos una gestión de código con servicios de integración continua o de cobertura de código.


\section{Motivación Personal}
\label{sec:mot}
Desde que empecé el instituto siempre me he decantado por una manera de trabajar más práctica que teórica. Si algo tengo que destacar de la carrera es el descubrir la programación, algo que sabía que era, pero que nunca había practicado, es decir, no había programado nunca un "Hola Mundo". \\
Una de las asignaturas que más me entusiasmo cursar, fue la de \emph{"Servicios y Aplicaciones Telemáticas"}, en la que desarrollamos aplicaciones web a través del \emph{framework} Django. En este proyecto además de un desarrollo web, con distintas tecnologías que utilice en dicha asignatura, hay otro objetivo, que es el más interesante para mi, que es seguir unas pautas de escribir código para proyectos de \emph{Software libre}.

\section{Objetivos}
\label{sec:objetivos}
El objetivo principal, por tanto, es la creación de un sitio web, para dar a conocer la librería python \emph{PyCardio}. Dicha librería ha sido desarrollada por varios alumnos y profesores de esta Universidad. El propósito de esta web es dar a conocer los módulos de la librería y mostrar su documentación. \\
Otro propósito de dicha web, al ser un proyecto de \emph{Software libre}, es facilitar la contribución al código de este paquete. Para alcanzar dichos fines, se abordarán las siguientes metas secundarias :
\begin{itemize}
    \item Publicar contenido mediante Jekyll.
    \item Crear la docWeb en GitHub Pages.
    \item Aprender a usar Liquid, para optimizar el desarrollo del contenido.
    \item Familiarizarse con el lenguaje de marcado \emph{ReStructuredText} para la generación automática de documentación de los módulos de \emph{PyCardio} mediante Sphinx.
    \item Saber distribuir paquetes de Python mediante módulos de Python como \emph{twine} o \emph{conda}.
    \item Gestionar el software medianto varios servicios como \emph{Travis CI}.
    \item Gestión de nuevas producciones del \emph{software} mediante \emph{releases}.
\end{itemize}

\section{Estructura de la memoria}
\label{sec:estruc}
Para finalizar la introducción, se explica como va a ir estructurada la memoria seguida de un breve resumen del contenido que trata cada sección:
\begin{enumerate}
    \item En el capítulo uno, como ya hemos visto, se introduce el por qué de este proyecto así como los objetivos que se tratan en él.
    \item En el segundo, \emph{Soluciones tecnológicas}, se explicarán las tecnologías usadas en el proyecto tanto en la generación del sitio web, como de gestión y documentación de proyectos Python.
    \item En el tercer capítulo, \emph{Gestión de código}, trataremos sobre la gestión de código en python y sobre el paquete que centramos el objetivo de este proyecto \emph{PyCardio}.
    \item En este cuarto capítulo, \emph{Propuesta del diseño web del proyecto}, presentaremos la web del proyecto donde se incluirá la documentación del módulo, guía de instalación, así como una guía para futuras contribuciones.
    \item En el quinto y último capítulo, \emph{Conclusiones y trabajo futuro}, expondremos las conclusiones sacadas así como el trabajo que se debería realizar tras la finalización de este mismo.
\end{enumerate}
