%%%%%%%%%%%%%%%%%%%%%%%%%%%%%%%%%%%%%%%%%%%%%%%%%%%%%%%%%%%%%%%%%%%%%%%%%%%%%%%
% CAPÍTULO 1 - INTRODUCCIÓN Y OBJETIVOS
%%%%%%%%%%%%%%%%%%%%%%%%%%%%%%%%%%%%%%%%%%%%%%%%%%%%%%%%%%%%%%%%%%%%%%%%%%%%%%%
\chapter{Introducción y objetivos}
\label{chap:intro}
\pagenumbering{arabic}
En este proyecto tratamos la importancia de seguir unas buenas pautas para la gestión y distribución de código. Donde de todas ellas ahondaremos en la creación de documentación web, donde nos ayudaremos de varias tecnologías. \\
El objetivo por tanto de este proyecto, es presentar las pautas a seguir para facilitar las contribuciones futuras en proyectos de software libre y mostrar una de ellas como, la documentación. \\
Por tanto, la intención de este capítulo es mostrar el contexto, la motivación que  me ha llevado para realizar dicho proyecto, objetivos de este, y la estructura que vamos a seguir para mostrar lo realizado.

\section{Contexto}
\label{sec:contex}
A los largos de los años, el software ha pasado por varias etapas en cuanto a su privatización. Antes del \emph{boom} de la informática, los que hacían uso de ella, compartían \emph{software} sin ninguna restricción, pero cuando llegaron los años 80 las compañías que vendían las computadoras comercializaban estas usando sistemas operativos privados, forzando al usuario a aceptar restricciones legales de modificar dicho \emph{software}.\\
Por estos años y debido a un error con un dispositivo, Richard Stallman fundó el proyecto GNU e introdujo la definición de \emph{software libre}. ¿Qué es software libre? Software libre es la cuestión de libertad de ejecutar, copiar, distribuir, estudiar, cambiar y mejorar el software.
\newpage
Es decir, significa que los usuarios de dicho software disponemos de las cuatro libertades esenciales:
\begin{itemize}
    \item Libertad 0, libertad para ejecutar el \emph{software} con cualquier propósito
    \item Libertad 1, acceso al código fuente; por lo que podemos modificar dicho \emph{software} para hacer lo que el usuario quiera
    \item Libertad 2, redistribuir copias
    \item Libertad 3, redistribución de versiones modificadas.
\end{itemize}

Y por tanto, uno de los objetivos de este proyecto es facilitar las contribuciones futuras para unos módulos sobre ingeniería biomédica desarrollados en la Universidad Rey Juan Carlos I, donde para ello, se creara una documentación web, usando tecnologías como Jekyll, que es un generador de contenido estático, o herramientas como ReadTheDocs o Sphinx, que facilitan la creación de documentación de \emph{software}


\section{Motivación Personal}
\label{sec:mot}
Desde que empecé el instituto siempre me he decantado por una manera de trabajar más práctica que teórica. Si algo tengo que destacar de la carrera es el descubrir la programación, algo que sabía que era, pero que nunca había practicado, es decir, no había programado nunca un "Hola Mundo". \\
Una de las asignaturas que más me entusiasmo cursar, fue la de \emph{"Servicios y Aplicaciones Telemáticas"}, en la que desarrollamos aplicaciones web a través del \emph{framework} Django. En este proyecto además de un desarrollo web, con distintas tecnologías que utilice en dicha asignatura, hay otro objetivo, que es el más interesante para mi, que es seguir unas pautas de escribir código para proyectos de \emph{Software libre}.

\section{Objetivos}
\label{sec:objetivos}
El objetivo principal, por tanto, es la creación de un sitio web, para dar a conocer la librería python \emph{PyCardio}. Dicha librería ha sido desarrollada por varios alumnos y profesores de esta Universidad. Donde con esta web lo que se propone es dar a conocer los módulos de la librería y mostrar su documentación. \\
Otro propósito de dicha web, al ser un proyecto de \emph{Software libre}, es facilitar la contribución al código de este paquete. Para alcanzar dichos fines, han sido abordados las siguientes metas secundarias :
\begin{itemize}
    \item Aprendizaje de nuevas tecnologías usadas en el proyecto.
    \item Publicar contenido mediante Jekyll.
    \item Crear la docWeb en GitHub Pages.
    \item Aprender a usar Liquid, para optimizar el desarrollo del contenido.
\end{itemize}

\section{Estructura de la memoria}
\label{sec:estruc}
Para finalizar la introducción, se explica como va a ir estructurada la memoria seguida de un breve resumen del contenido que trata cada sección:
\begin{enumerate}
    \item En el capítulo uno, como ya hemos visto, se introduce el por qué de este proyecto así como los objetivos que se tratan en él.
    \item En el segundo, \emph{Soluciones tecnológicas}, se explican las tecnologías usadas en el proyecto tanto en la generación del sitio web, como de gestión y documentación de proyectos Python.
    \item En el tercer capítulo, \emph{Gestión de código}, trataremos sobre la gestión de código en python y sobre el paquete que centramos el objetivo de este proyecto \emph{PyCardio}.
    \item En este cuarto capítulo, \emph{Propuesta del diseño web del proyecto}, presentamos la web del proyecto donde se incluirá la documentación del módulo, guía de instalación, así como una guía para futuras contribuciones.
    \item En el quinto y último capítulo, \emph{Conclusiones y trabajo futuro}, exponemos las conclusiones sacadas así como el trabajo que se debería realizar tras la finalización de este mismo.
\end{enumerate}
